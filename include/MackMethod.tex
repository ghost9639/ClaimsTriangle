% Created 2025-10-23 Thu 20:57
% Intended LaTeX compiler: pdflatex
\documentclass[a4paper]{article}
\usepackage[utf8]{inputenc}
\usepackage[T1]{fontenc}
\usepackage{graphicx}
\usepackage{longtable}
\usepackage{wrapfig}
\usepackage{rotating}
\usepackage[normalem]{ulem}
\usepackage{amsmath}
\usepackage{amssymb}
\usepackage{capt-of}
\usepackage{hyperref}
\usepackage{times}
\usepackage{amsmath}
\usepackage[utf8]{inputenc}
\usepackage{natbib}
\usepackage{graphicx}
\author{Rahul Ghosh}
\date{\today}
\title{Mack Method Assumptions}
\hypersetup{
 pdfauthor={Rahul Ghosh},
 pdftitle={Mack Method Assumptions},
 pdfkeywords={},
 pdfsubject={},
 pdfcreator={},
 pdflang={English}}
\begin{document}

\maketitle
\end{titlepage}

\tableofcontents
\flushleft

Suppose, for a regular claims triangle, we have a system as such:

\begin{center}
\begin{tabular}{rrrrr}
Year & 1 year claims & 2 year claims & 3 year claims & 4 year claims\\
\hline
2000 & 14124 & 10243 & 80425 & 62041\\
2001 & 15083 & 12421 & 91240 & \\
2002 & 13132 & 11023 &  & \\
2003 & 15850 &  &  & \\
\end{tabular}
\end{center}

We can generalise this as

\begin{center}
\begin{tabular}{rllll}
Year & 1 year claims & 2 year claims & 3 year claims & 4 year claims\\
\hline
2000 & C\textsubscript{1,1} & C\textsubscript{1,2} & C\textsubscript{1,3} & C\textsubscript{1,4}\\
2001 & C\textsubscript{2,1} & C\textsubscript{2,2} & C\textsubscript{2,3} & \\
2002 & C\textsubscript{3,1}\} & C\textsubscript{3,2} &  & \\
2003 & C\textsubscript{4,1} &  &  & \\
\end{tabular}
\end{center}

\(C_{i,j}\) is used to model claims, with \(R_i\) being the outstanding claims in oncoming years. Then we can search for \(\sum^n_{i=1} R_i\) as the claims reserve.
\end{document}
